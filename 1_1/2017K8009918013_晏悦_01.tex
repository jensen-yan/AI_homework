\documentclass{ctexart}
\usepackage[utf8]{inputenc}

\title{AI-homework1}
\author{晏悦 2017K8009918013}
\date{August 30 2019}

\begin{document}

\maketitle
证明Gain(X,Y) = H(Y) - H(Y,X)>0.

证明:

由于先获取X的信息熵,再获取基于X的Y的信息熵,等价于先获取Y的信息熵,再获取X的信息熵,所以:
\[H(X,Y) = H(X) + H(X|Y) = H(Y) + H(Y|X)\]

Gain(X,Y) = H(Y) - H(Y|X) = \Sigma_y p(y)*log(p(y)) + \Sigma_x p(x)\Sigma _y p(y|x)log(p(y|x))

=H(X) - H(X|Y) = \Sigma_x p(x)*log(p(x)) + \Sigma_y p(y)\Sigma _x p(x|y)log(p(x|y))

=H(X) + H(Y) - H(X,Y) = \Sigma_x\Sigma_y p(x,y)*(-log(p(x) - log(p(y) + log(p(x,y)))

=\Sigma_x\Sigma_y p(x,y)*log(\frac{p(x,y)}{p(x)p(y)})

=D(P(X,Y)||P(X)P(Y))

其中D是KL散度公式,那么只用证明KL散度大于零即可

D(P||Q) = \Sigma_x p(x)log(\frac{p(x)}{q(x)})

>=-log(\Sigma_x p(x)*\frac{q(x)}{p(x)})

(这一步由Jensen不等式得到)

=0

所以证明完毕,信息熵增量大于等于0,其中为0时,当且仅当X,Y独立。
\end{document}
