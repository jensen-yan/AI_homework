\documentclass{ctexart}
\usepackage[utf8]{inputenc}

\title{AI-homework3}
\author{晏悦 2017K8009918013}
\date{September 2019}

\begin{document}

\maketitle

请证明$\frac{1}{N}\Sigma_{i=1}^N I(f(x_i) \not= y_i) \leq exp(-2\Sigma_{k=1}^K\gamma_k^2)$,其中$\gamma_k=\frac{1}{2}-e_k$

证明:

1.首先证明$\Pi_{k=1}^K\sqrt{1-4\gamma_k^2}\leq \exp(-2\Sigma_{k=1}^K \gamma_k^2)$

不妨设$x_k = \gamma_k^2$,由于$\gamma_k\in(0,\frac{1}{2})$,所以$x_k\in(0,\frac{1}{4})$.

两边同时去对数ln

得到$\Sigma_{k=1}^K \sqrt{1-4x_k} \leq -2\Sigma x_k$

即证$f(x) = \sqrt{1-4x}+2x \leq 0$ 对x恒成立。

对函数求导,导函数恒<0,$f(x)\leq f(0) = 0$成立



2.接下来证明$\frac{1}{N}\Sigma_{i=1}^N I(f(x_i) \not= y_i) \leq \Pi_{k=1}^K\sqrt{1-4\gamma_k^2}$

显然$I(f(x_i)\not=y_i) < \exp(-y_i*f(x_i))$

并且由于$f(x_i)=sign(\Sigma_k\alpha_k*f_k(x_i))$

所以

$\frac{1}{N} \Sigma_{i=1}^N I(f(x_i) \not= y_i)$ 

$<\frac{1}{N}  \Sigma_{i=1}^N  \exp(-y_i f(x_i))$

$=\frac{1}{N}\Sigma_{i=1}^N \exp(-y_i \Sigma_k\alpha_k*f_k(x_i))$

$=\frac{1}{N}\Sigma_{i=1}^N \Pi_k\exp(-y_i \alpha_k*f_k(x_i)$

由于$d_{0,i} = \frac{1}{N}$,且$d_{k-1,i} \exp(-\alpha_k y_i f_k(x_i))= Z_k d_{k,i}$

所以上式等于

$=Z_1 \Pi_{k=2}^K \exp(-y_i \alpha_k*f_k(x_i)$

$=Z_1 Z_2 \Pi_{k=3}^K \exp(-y_i \alpha_k*f_k(x_i)$

$=\Pi_{k=1}^K Z_k$

又因为$d_{k-1,i} \exp(-\alpha_k y_i f_k(x_i))= Z_k d_{k,i}$,其中$d_{k,i}$满足归一化条件

左右两边对i求和

得到$Z_k = \Sigma_id_{k-1,i} \exp(-\alpha_k y_i f_k(x_i))$

$=\Sigma_{y_i = f_k(x_i)} d_{k-1,i} \exp(-\alpha_k ) + \Sigma_{y_i \not= f_k(x_i)} d_{k-1,i} \exp(\alpha_k)$

$=(1-e_k) e^{-\alpha_k} + e_k e^{\alpha_k}$

其中$e_k = \Sigma_i d_{k-1,i} (y_i \not= f_k(x_i))$

$=2\sqrt{e_k(1-e_k)}$

$=\sqrt{1-4\gamma_k}$

两边同时对k求连续乘,得到要证明的不等式,证明完毕!


\end{document}
